% !TEX program = lualatex
% !TEX options = -shell-escape -synctex=1 -interaction=nonstopmode -file-line-error "%DOC%"
% !BIB program = bibtex

\documentclass[12pt, oneside]{book}

\usepackage[a-1b]{pdfx}
\hypersetup{hidelinks, bookmarksnumbered}
\usepackage{tocbibind}

\usepackage[top = 1in, right = 1in, bottom = 1in, left = 1.5in]{geometry}
\usepackage[doublespacing]{setspace}

\pagestyle{plain}

\usepackage{fontspec, unicode-math}
\setmainfont{XCharter}
\setmonofont{FiraMono}[Scale = 0.9]
\setmathfont{TeX Gyre Pagella Math}

\usepackage{minted}
\usemintedstyle{vs}
\setminted{fontsize = \small, baselinestretch = 1}
\setmintedinline{fontsize = \normalsize}

\begin{document}

\frontmatter

\begin{center}
\begin{singlespace}
\vspace*{0.5in}

\textbf{\uppercase{Yocto-CFA}}

\vspace*{1in}

by

Leandro Facchinetti

\vspace*{1.5in}

A dissertation submitted to Johns Hopkins University\\in conformity with the requirements for the degree of Doctor of Philosophy

\vspace*{0.5in}

Baltimore, Maryland

August 2020
\end{singlespace}
\end{center}

\thispagestyle{empty}
\clearpage

\chapter{Abstract}

% TODO

\paragraph{Primary Reader and Advisor:}

Dr.~Scott Fraser Smith.

\paragraph{Readers:}

Dr.~Zachary Eli Palmer and Dr.~Matthew Daniel Green.

\chapter{Acknowledgements}

% TODO

\tableofcontents
\listoftables
\listoffigures

\mainmatter

% TODO: Introduction

\chapter{Developing an Analyzer}

% TODO: An overview of the rest of the section

\section{The Analyzed Language: Yocto-JavaScript}

Our first decision when developing an analyzer is which language it should analyze. This decision is important because the analyzed language may affect the precision and the running time of the analyzer. For example, there is an analysis technique called \(k\)-CFA~\cite{k-cfa} that may be slower when applied to languages with higher-order functions than when applied to languages with objects (the algorithmic complexity of the former is exponential and the latter is polynomial)~\cite{m-cfa}.

In this dissertation we are interested in analysis techniques for higher-order functions. Fortunately, most popular languages support them: JavaScript, Java, Python, and so forth. From all these languages, we choose JavaScript because it is the most widely used programming language~\cite{stack-overflow-developer-survey, jet-brains-developer-survey} and because it is understood by most people, even those who do not use it regularly, so it makes for a good language of discourse.

Unfortunately, JavaScript is a big language with many features besides higher-order functions, and if we tried to support all of them we would end up with an analyzer that is too complex. Instead, we choose to analyze only a subset of JavaScript features, namely the subset related to higher-order functions: we call it \emph{Yocto-JavaScript} (that is \(\mathrm{JavaScript} \times 10^{-24}\)).

\paragraph{Values in Yocto-JavaScript.}

JavaScript has many kinds of values: strings (for example, \mintinline{js}{"Leandro"}), numbers (for example, \mintinline{js}{29}), arrays (for example, \mintinline{js}{["Leandro", 29]}), objects (for example, \mintinline{js}!{ name: "Leandro", age: 29 }!), and so forth. Yocto-JavaScript has only one kind of value: functions. Yocto-JavaScript functions are written as something called \emph{arrow function expressions}~\cite{arrow-function-expressions}, for example, \mintinline{js}{x => x} is a function with one parameter called \mintinline{js}{x} (before the \mintinline{js}{=>}) and a body which consists of just a reference to the variable \mintinline{js}{x} (after the \mintinline{js}{=>}). Yocto-JavaScript functions return the result of computing their body, so our example function simply returns whatever argument it was passed. Yocto-JavaScript functions are limited to a single argument.

\paragraph{Operations in Yocto-JavaScript.}

JavaScript has many operations: numbers can be added together (for example, \mintinline{js}{29 + 1}), objects can have their properties accessed (for example, \mintinline{js}!{ name: "Leandro", age: 29 }.name!), and so forth. Yocto-JavaScript has only two operations: functions can be called and variables can be referenced.

The following is an example of a function call:

\begin{minted}{js}
(x => x)(y => y)
\end{minted}

The function being called is \mintinline{js}{x => x}, and the argument being passed to it is \mintinline{js}{y => y}—remember that functions are the only kind of value in Yocto-JavaScript, so the argument must be a function. As discussed above the \mintinline{js}{x => x} function simply returns whatever argument it was passed, so the result of the program is \mintinline{js}{y => y}.

The program above also includes examples of variable references: the \mintinline{js}{x} and \mintinline{js}{y} to the right of the respective \mintinline{js}{=>}.

Parentheses indicate the precedence of the operations, for example, if \mintinline{js}{f}, \mintinline{js}{g}, and \mintinline{js}{h} are functions, then \mintinline{js}{f(g)(h)} means \mintinline{js}{f} gets called first with \mintinline{js}{g} as argument, and the result is called with \mintinline{js}{g} as argument

% TODO: Grammar.
% TODO: Expressive power.

\appendix

% TODO

\backmatter

\bibliographystyle{plain}
\bibliography{\jobname}

\chapter{Biographical Statement}

% TODO

\end{document}
